\documentclass[letterpaper, 10 pt, conference]{ieeeconf}  % Comment this line out if you need a4paper

%\documentclass[a4paper, 10pt, conference]{ieeeconf}      % Use this line for a4 paper

%\IEEEoverridecommandlockouts                              % This command is only needed if 
                                                          % you want to use the \thanks command

%\overrideIEEEmargins                                      % Needed to meet printer requirements.

% See the \addtolength command later in the file to balance the column lengths
% on the last page of the document

% The following packages can be found on http:\\www.ctan.org
\usepackage{graphics} % for pdf, bitmapped graphics files
\usepackage{epsfig} % for postscript graphics files
\usepackage{graphicx}
\usepackage{mathptmx} % assumes new font selection scheme installed
\usepackage{times} % assumes new font selection scheme installed
\usepackage{mathtools} % assumes amsmath package installed
\usepackage{amssymb}  % assumes amsmath package installed
\usepackage{tikz}
\usepackage{tabulary}
\newcommand{\hvec}{\overset{\rightharpoonup}}
\newcommand{\argmin}{\arg\!\min}
\newcommand{\norm}[1]{\left\lVert#1\right\rVert}
\newcommand{\quotes}[1]{``#1''}
\usetikzlibrary{calc,positioning, fit, arrows}
%\usepackage{biber}

\makeatletter
\newenvironment{tablehere}
  {\def\@captype{table}}
  {}

\newenvironment{figurehere}
  {\def\@captype{figure}}
  {}
\makeatother

\newcommand{\vect}[1]{\ensuremath{\mathbf{#1}}}
\newcommand{\mat}[1]{\ensuremath{\mathbf{#1}}}
\newcommand{\transpose}{\ensuremath{\mathsf{T}}}
\newcommand{\of}[1]{\ensuremath{\left(#1\right)}}

\title{\LARGE \bf
Multi-dimensional Optimization of Gasoline-fueled Variable Pitch Multirotor Aircraft
}


\author{Dallin Briggs, Gary Ellingson% <-this % stops a space
%\thanks{$^{1}$James Jackson is a MS student in the Department of Mechanical Engineering, Brigham Young University
%        {\tt\small jamesjackson@byu.edu}}%
%\thanks{$^{2}$Gary Ellingson is a MS student in the Department of Mechanical Engineering, Brigham Young University
%        {\tt\small gary.ellingson@byu.edu}}%
}


\begin{document}



\maketitle
\thispagestyle{empty}
\pagestyle{empty}


%%%%%%%%%%%%%%%%%%%%%%%%%%%%%%%%%%%%%%%%%%%%%%%%%%%%%%%%%%%%%%%%%%%%%%%%%%%%%%%%
\begin{abstract}

Currently, there are numerous multirotor UAVs that are available commercially and to consumers that are capable of lifting small payloads for an endurance of approximately 30 minutes. While these current multirotors are good for short endurance missions, there are few options for larger, higher payload capacity multirotors that are able to maintain flight for longer than an hour. This paper describes the optimization process used to develop a multirotor platform that maximizes the flight time of a gasoline-fueled multirotor UAV by varying several design variables related to the power required and the aerodynamics of the model.

\end{abstract}


%%%%%%%%%%%%%%%%%%%%%%%%%%%%%%%%%%%%%%%%%%%%%%%%%%%%%%%%%%%%%%%%%%%%%%%%%%%%%%%%
\section{INTRODUCTION}

\begin{itemize}
	\item{This paper looks at the why and how we are going to optimize a multirotor platform to maximize flight time.}
	\item{Talk about how the variables that are going to be optimized are rotor radius, fuel consumption rate, number of blade per rotor, number of rotors, rotor speed, and rotor pitch.}
	\item{Talk about why these variables were selected as design variables.}
	\item{Talk about some physical constraints, especially those imposed by the FAA.}
\end{itemize}


%%%%%%%%%%%%%%%%%%%%%%%%%%%%%%%%%%%%%%%%%%%%%%%%%%%%%%%%%%%%%%%%%%%%%%%%%%%%%%%
\section{Background}

Control of variable pitch multirotor aircraft was studied by Cutler \cite{Cutler2012}. There is a small, electric variable pitch quad commercially available \cite{stingray2016}.  There are also several hobbyist attempts at gas powered variable pitch quads \cite{diy2016}, \cite{hackaday2016}.

In our knowledge there has been no attempt to use a gas powered variable pitch quad to maximize the flight time for carrying a relatively large payload. 

There are many commercially available parts for a gas quad, including engines and variable pitch helicopter propellers. 

\section{MOTIVATION}

All commercial and consumer multirotor UAVs are powered by Lithium-Polymer batteries because of their high energy capacity to weight ratios. Although these batteries are very good and reliable, they still offer much lower specific energy ratios then that of liquid fuels such as gasoline. The average specific energy for Lithium-Polymer batteries is up to 0.95 MJ/kg, where gasoline is 46.4 MJ/kg. This means that a gasoline-fueled multirotor can potential fly much longer than a battery powered multirotor because it can carry more energy onboard. 

\begin{itemize}
	\item{Will talk more about some of the details using a gas engine entails in the paper.}
	\item{Speak briefly on the reasons for requiring variable pitch with a gas engine.} 
	\item{Speak on current platforms and capabilities and explain how this could fundamentally change the way multirotors are used in commercial applications, but not really for consumers.}
\end{itemize}


\begin{figurehere}
	\begin{center}
		\includegraphics[width=.40\textwidth]{current_capabilities.jpg}
		\caption{\textit{Graphic showing current platforms and how our would be better.}}
		\label{current_cap}
	\end{center}
\end{figurehere}

	
%%%%%%%%%%%%%%%%%%%%%%%%%%%%%%%%%%%%%%%%%%%%%%%%%%%%%%%%%%%%%%%%%%%%%%%%%%%%%%%%%%
\section{METHODS}

Because the quad is going to be build using commercially available off-the-shelf parts the optimization must be closely based on reality.  This means we only can pick the size of engine that we are going to use.  All the other data about the engine (mass, maximum power, etc.) have to be a function of displacement size.  We collected data about popular two stroke RC aircraft engines from several sources to produce the below plots. 

\begin{figurehere}
	\includegraphics[width=0.5\textwidth]{mass.png}
	\caption{Stand in figure showing the data used to obtain the empirical based surrogate model of engine mass as a function of displacement size.}
		\label{fig:mass}
\end{figurehere}

\begin{figurehere}
	\includegraphics[width=0.5\textwidth]{max_power.png}
	\caption{Stand in figure showing the data used to obtain the empirical based surrogate model of engine maximum power as a function of displacement size.}
		\label{fig:power}
\end{figurehere}

The engine efficiency (friction and heat transfer losses), however, is a function of power capability of the engine (max power) and the power engine is required to produce.  For this we will need a better model.  \textit{We are still working on this. Since efficiency numbers for RC engines aren't publicly available, we need a scientific model.  For now we just assume a constant efficiency. We plan to work on this in the future.} 

Although the off-the-shelf constraint also applies to choosing the rotor parameters, continuous variables approximating the rotor specs are used and then analyzed as follows. 

Much of the work done in the methodology of optimizing flight time was to find adequate models for the thrust, torque, and power required from the main rotor disks.

First model used in the beginning of the project used a simplified version of Momentum theory and actuator disk theory from \cite{helicopters2016}. This model took into account the number of rotors, the rotor radius, the disc loading of each rotor, and how much fuel was available onboard the aircraft. However, this model did not take into account the blade speed, blade pitch angle, or the number of blades per rotor. The model did, however, give us a baseline for initial optimization results.

Second model used to develop the optimization framework for this project was based on an MIT lecture given on aircraft propeller performance given in \cite{mit2016}. In order to test the theory of this model and also to further the progress of the optimization model, only two of the design variables were chosen to optimize: rotor radius and fuel consumption rate, $f_{rate}$. 

To properly analyses the system, power required ($P_{req}$) and thrust produced ($T_{prod}$) have to be calculated to make sure that the rotor radius had a relationship to the fuel consumption rate. These calculations used some approximations in the rotor dynamics. The following calculations come from a MIT lecture on thermodynamics.

In order to calculate thrust of each rotor blade, the thrust has to be calculated differentially along the blade because of the angular velocity. The differential thrust is given by 
\begin{equation}
	dT = dLcos(\phi + \alpha_i) - dDsin(\phi + \alpha_i)
\end{equation}

where $\phi = \tan^{-1}(\frac{V_i}{V_n})$, and $\alpha_i = \theta - \tan^{-1}(\frac{V_i}{V_n})$ and $\theta$ is the blade pitch angle (held constant in this case of the optimization). $dL$ and $dD$ are the differentials of the lift and drag, respectively, and are given by
\begin{equation}
	dL = \frac{1}{2}\rho V_e^2 c C_l dr
\end{equation}

and 

\begin{equation}
	dD = \frac{1}{2}\rho V_e^2 c C_d dr.
\end{equation}

The coefficients of lift and drag were held constant at $C_l = 0.5438$ and $C_d = 0.0141$ (obtained from a symmetric airfoil pitch at $\alpha = 5$ degrees). Splitting the rotor into 10 sections and integrating seemed to give a reasonable estimate to the thrust produced by each rotor. This was then constrained by making sure the thrust produced was greater than or equal to the thrust required, which was
\begin{equation}
	T_{req} = AUW\times g.
\end{equation}

In order to make sure that enough power was being produced by the engine (and prevent the fuel consumption rate to going to 0), the power produced was given by 
\begin{equation}
	P_{prod} = f_{rate}E_{\rho f} \eta_{engine}
\end{equation}

and was made sure that it was greater than or equal to the power required. To calculate the power required, the torque required to turn each rotor was calculated differentially by the equation
\begin{equation}
	dQ = r[dLsin(\phi + \alpha_i) + dDcos(\phi + \alpha_i)].
\end{equation}

The power required was then calculated by 
\begin{equation}
	P_{req} = \omega_rQ_{req}2n_{rotors}.
\end{equation}

This model did converge to an optimal solution when executed with gradient based optimization methods discussed in the next section. However, in this model, not all of the design variables were implemented due to the complexity these additional variables brought to the calculus in this method.

The third model used to represent the aerodynamics and thrust of each rotor blade was from \cite{chen1979simplified}, a NASA model used in helicopter simulations several years ago. Instead of trying to implement blade element theory across the blade, this new model is able to approximate the thrust and torque required for the rotor given numerous variables. Instead of only varying rotor radius and fuel consumption rate, like last homework's model, this model attempts to optimize given rotor radius, fuel consumption rate, number of blade per rotor, number of rotors, rotor speed, and rotor pitch. The thrust, T, is given by equation \ref{thrust} and the torque, Q, is represented by equations \ref{torque} and \ref{torquecoeff}.

\begin{multline}
T = \frac{N}{2} \rho a c R (\omega R)^2 (.5 (1-\epsilon^2) \lambda \\
 + \theta_0 (\frac{1}{3} + \frac{\mu^2}{2} (1-\epsilon)) + \theta_t (.25+\frac{\mu^2}{2} (1-\epsilon^2))\\
- \frac{\mu}{2} (1-\epsilon^2) (B_{lc}-K_l b_l) - a_0 (\frac{1}{3} +\frac{\mu^2}{2} (1-\epsilon)) K_l \\
+ a_l (\frac{\mu}{2} \epsilon (1-\epsilon)) - \frac{\dot{a_0}}{\omega} (\frac{1}{3} -\frac{\epsilon}{2}) + \frac{\dot{b_l}}{\omega} (\frac{\mu}{4} (1-\epsilon)^2) \\
+ \frac{\mu}{4} (1-\epsilon^2) (\frac{p}{\omega cos(\beta_w)} + \frac{q}{\omega sin(\beta_w))} ) - N M_\beta \ddot{a_0}
\label{thrust}
\end{multline}
and
\begin{equation}
Q = N/2 \rho a c R^2 (\omega R)^2 (Q_{coeff})
\label{torque}
\end{equation}
where
\begin{multline}
Q_{coeff} = \delta/(4 a) (1 + (1-\epsilon^2) \mu^2) \\
 - (\theta_0 - K_l a_0)(\lambda/3+(\epsilon/3-.25) \dot{a_0}/\omega + \mu/6 \dot{b_l}/\omega\\  
- \mu \epsilon/4 (\dot{b_l}/\omega-a_l) + \mu/6 (p/\omega cos(\beta_w) +q/\omega sin(\beta_w))) \\
- (A_{lc}-K_l a_l) ((.125-\epsilon/6) (\dot{a_l}/\omega+b_l)-\mu/6 a_0 \\
+b_l/16 (1-\epsilon^2) \mu^2+1/8 (-p/\omega sin(\beta_w)+q/\omega cos(\beta_w)))\\ 
- (B_lc-K_l b_l) ((.125-\epsilon/6) (\dot{b_l}/\omega-a_l)\\
+(\epsilon/4-1/6) \mu \dot{a_0}/\omega+.5 (1-\epsilon^2) (\mu \lambda/2+a_l/8 \mu^2)\\ 
+ 1/8 (p/\omega cos(\beta_w)+q/\omega sin(\beta_w))) - \theta_t (\lambda/4\\ 
+ (\epsilon/4-1/5) \dot{a_0}/\omega+\mu/8 (\dot{b_l}/\omega)-\epsilon \mu/6 (\dot{b_l}/\omega-a_l))\\  
- .5 (1-\epsilon^2) (\lambda^2+\lambda \mu a_l+2 \lambda \epsilon \dot{a_0}/\omega\\  
+\mu \epsilon (a_l \dot{a_0}/\omega+a_0 (\dot{a_l}/\omega+b_l))\\  
+ \mu^2 (a_0^2/2+3/8 a_l^2+1/8 b_l^2)) + \mu/3 (a_l (\dot{a_0}/\omega) \\
+ a_0 (\dot{a_l}/\omega+b_l))+2/3 \lambda (\dot{a_l}/\omega)\\  
- (-\mu/3 a_0 + (.25-\epsilon/3) (\dot{a_l}/\omega+b_l)) \\ 
(-p/\omega sin(\beta_w)+q/\omega cos(\beta_w))\\ 
- (.25-\epsilon/3) (\dot{b_l}/\omega-a_l) (p/\omega cos(\beta_w)\\ 
+ q/\omega sin(\beta_w)) - .125 (-p/\omega sin(\beta_w)+q/\omega cos(\beta_w))^2 \\ 
-.125 (p/\omega cos(\beta_w)+q/\omega sin(\beta_w))^2 -  \\
(.25-2/3 \epsilon+\epsilon^2/2) ((\dot{a_0}/\omega)^2 \\ 
+ .5 ((\dot{a_l}/\omega+b_l)^2+(\dot{b_l}/\omega-a_l)^2))
\label{torquecoeff}
\end{multline}

Please see \cite{chen1979simplified} for definition of terms. Although this model takes into account all of the design variables chosen for this project, it is difficult to optimize because of the complexity of the functions. \textit{Currently, this model fails to converge to a solution, so we are working with it to find the source of the issue.}


\section{OPTIMIZATION}

Our design variables are rotor radius, fuel consumption rate, number of blades per rotor, number of rotors, rotor speed, and rotor pitch. Rotor radius and number of blades per rotor, and number of rotors were chosen because this is an easy parameter to vary on multirotors due to the availability of numerous rotor blades available for small rotor aircraft. These variables also have a large influence on the flight time of the multirotor. Rotor speed and rotor pitch were chosen because these variables are easy to change on the physical system through selecting the proper gearing ratios. Fuel consumption rate seems like an odd variable for an optimizer because intuitively an optimizer would just minimize this value to maximize the flight time; however, fuel consumption rate is constrained by the size of engine required to produce the amount of power needed to provide enough lift for the multirotor. All of these design variables, however, have bound constraints that keep them in a reasonable space of commercially available parts/performance.

Setup the formal optimization problem
\begin{equation}
min. \quad f\of{x} = -t_{flight}\of{x}
\label{eq:objective}
\end{equation}
\begin{equation}
w.r.t. \quad x = [variables]^T
\label{eq:vars}
\end{equation}
\begin{equation}
s.t. \quad cons\of{x} \leq 0 
\label{eq:constrants}
\end{equation}

Flight time was calculated by 
\begin{equation}
	t_{flight} = K_{energy}*f_{capacity}/f_{rate}.
\end{equation}
Without constraints, the solution to maximizing flight time would be trivial to calculate as well as impossible to construct. Therefore to optimize the system we must use constraints that ensure the quad is feasible. The constrains are on the power produced by the engine ($P_{prod}$), power required by the prop ($P_{req}$), thrust produced by the prop ($T_{prod}$), lift required by the quad ($T_{req}$) and maximum power capabilities of the engine ($P_{max}$). Where $P_{preq} \leq P_{prod}$, $T_{req} \leq T_{prod}$, and $P_{prod} \leq P_{max}$.

We then will discuss the following:
\begin{itemize}
	\item{scaling of the design variables}
	\item{how we got gradients}
	\item{the optimization method that we utilized}
	\item{python vs matlab benefits and challenges}
\end{itemize}

Since we can only optimize for maximum flight time give a payload (parameter) we can further explore how the characteristics of a gas quad change as we optimize the flight time for different payloads. It may turn out that the optimal quad parameters change greatly as it is optimized for different payload.  In this case we could choose several payloads and optimize over all of then together. \textit{We will present these results with a figure and some discussion.}

\begin{figurehere}
	\includegraphics[width=0.5\textwidth]{payload.jpg}
	\caption{Stand in figure showing optimal flight time as a function of payload.}
		\label{fig:payload}
\end{figurehere}

%Here we can do some discussion on constraint sensitivity - aerodynamics, engine efficiencies...

\section{RESULTS AND DISCUSSION}

We have found that the flight time of a gas-quad can be maximized by constructing with the following design.

\begin{tablehere}
\centering
\begin{tabulary}{0.5\textwidth}{|C|C|C|C|C|C|}
\hline
       & displacement & fuel rate & rotor radius & other \\ \hline
  & 100 & 30 & 250 & 150 \\ \hline
\end{tabulary}
\caption{Optimal design variables}
\label{table:fw_loop_rates}
\end{tablehere}


%%%%%%%%%%%%%%%%%%%%%%%%%%%%%%%%%%%%%%%%%%%%%%%%%%%%%%%%%%%%%%%%%%%%%%%%%%%%%%%%%%

\section{CONCLUSION}

lots of really good conclusions

%%%%%%%%%%%%%%%%%%%%%%%%%%%%%%%%%%%%%%%%%%%%%%%%%%%%%%%%%%%%%%%%%%%%%%%%%%%%%%%%
% \section*{APPENDIX}

% Appendixes should appear before the acknowledgment.

% \section*{ACKNOWLEDGMENT}

% Important people/organizations who made it possible


%%%%%%%%%%%%%%%%%%%%%%%%%%%%%%%%%%%%%%%%%%%%%%%%%%%%%%%%%%%%%%%%%%%%%%%%%%%%%%%%

\bibliography{./library}
\bibliographystyle{ieeetr}



\end{document}


%%%%%%%%%%%%%%%%%%%%%%%%%%%%%%%%%%%%%%%%%%%%%%%%%%%%%%%%%%%%%%%%%%%%%%%%%%%%%

% SAVED STUFF






%new document




