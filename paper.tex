\documentclass[letterpaper, 10 pt, conference]{ieeeconf}  % Comment this line out if you need a4paper

%\documentclass[a4paper, 10pt, conference]{ieeeconf}      % Use this line for a4 paper

%\IEEEoverridecommandlockouts                              % This command is only needed if 
                                                          % you want to use the \thanks command

%\overrideIEEEmargins                                      % Needed to meet printer requirements.

% See the \addtolength command later in the file to balance the column lengths
% on the last page of the document

% The following packages can be found on http:\\www.ctan.org
\usepackage{graphics} % for pdf, bitmapped graphics files
\usepackage{epsfig} % for postscript graphics files
\usepackage{graphicx}
\usepackage{mathptmx} % assumes new font selection scheme installed
\usepackage{times} % assumes new font selection scheme installed
\usepackage{mathtools} % assumes amsmath package installed
\usepackage{amssymb}  % assumes amsmath package installed
\usepackage{tikz}
\usepackage{tabulary}
\newcommand{\hvec}{\overset{\rightharpoonup}}
\newcommand{\argmin}{\arg\!\min}
\newcommand{\norm}[1]{\left\lVert#1\right\rVert}
\newcommand{\quotes}[1]{``#1''}
\usetikzlibrary{calc,positioning, fit, arrows}
%\usepackage{biber}

\makeatletter
\newenvironment{tablehere}
  {\def\@captype{table}}
  {}

\newenvironment{figurehere}
  {\def\@captype{figure}}
  {}
\makeatother

\newcommand{\vect}[1]{\ensuremath{\mathbf{#1}}}
\newcommand{\mat}[1]{\ensuremath{\mathbf{#1}}}
\newcommand{\transpose}{\ensuremath{\mathsf{T}}}
\newcommand{\of}[1]{\ensuremath{\left(#1\right)}}

\title{\LARGE \bf
Multi-dimensional Optimization of Gasoline-fueled Variable Pitch Multirotor Aircraft
}


\author{Dallin Briggs, Gary Ellingson% <-this % stops a space
%\thanks{$^{1}$James Jackson is a MS student in the Department of Mechanical Engineering, Brigham Young University
%        {\tt\small jamesjackson@byu.edu}}%
%\thanks{$^{2}$Gary Ellingson is a MS student in the Department of Mechanical Engineering, Brigham Young University
%        {\tt\small gary.ellingson@byu.edu}}%
}


\begin{document}



\maketitle
\thispagestyle{empty}
\pagestyle{empty}


%%%%%%%%%%%%%%%%%%%%%%%%%%%%%%%%%%%%%%%%%%%%%%%%%%%%%%%%%%%%%%%%%%%%%%%%%%%%%%%%
\begin{abstract}

Currently, there are numerous multirotor UAVs that are available commercially and to consumers that are capable of lifting small payloads for an endurance of approximately 30 minutes. While these current multirotors are good for short endurance missions, there are few options for larger, higher payload capacity multirotors that are able to maintain flight for longer than an hour. This paper describes the optimization process used to develop a multirotor platform that maximizes the flight time of a gasoline-fueled multirotor UAV by varying several design variables related to the power required and the aerodynamics of the model.

\end{abstract}


%%%%%%%%%%%%%%%%%%%%%%%%%%%%%%%%%%%%%%%%%%%%%%%%%%%%%%%%%%%%%%%%%%%%%%%%%%%%%%%%
\section{INTRODUCTION}

\begin{itemize}
	\item{This paper looks at the why and how we are going to optimize a multirotor platform to maximize flight time.}
	\item{Talk about how the variables that are going to be optimized are rotor radius, fuel consumption rate, number of blade per rotor, number of rotors, rotor speed, and rotor pitch.}
	\item{Talk about why these variables were selected as design variables.}
	\item{Talk about some physical constraints, especially those imposed by the FAA.}
\end{itemize}


%%%%%%%%%%%%%%%%%%%%%%%%%%%%%%%%%%%%%%%%%%%%%%%%%%%%%%%%%%%%%%%%%%%%%%%%%%%%%%%
\section{Background}

Control of variable pitch multirotor aircraft was studied by Cutler \cite{Cutler2012}. There is a variable pitch quad commercially available \cite{stingray2016}.  There are also several hobbyist attempts at gas powered variable pitch quads \cite{diy2016}, \cite{hackaday2016}.

In our knowledge there has been no attempt to use a gas powered variable pitch quad to maximize the flight time for carrying a large payload. 

There are many commercially available parts for a gas quad, including motors 

\section{MOTIVATION}

Almost all commercial and consumer multirotor UAVs are powered by Lithium-Polymer batteries because of their high energy capacity to weight ratios. Although these batteries are very good and reliable, they still offer much lower specific energy ratios then that of liquid fuels such as gasoline. The average specific energy for Lithium-Polymer batteries is up to 0.95 MJ/kg, where gasoline is 46.4 MJ/kg. This means that a gasoline-fueled multirotor can potential fly much longer than a battery powered multirotor because it can carry more energy onboard. 

\begin{itemize}
	\item{Will talk more about some of the details using a gas engine entails in the paper.}
	\item{Speak briefly on the reasons for requiring variable pitch with a gas engine.} 
	\item{Speak on current platforms and capabilities and explain how this could fundamentally change the way multirotors are used in commercial applications, but not really for consumers.}
\end{itemize}


\begin{figurehere}
	\begin{center}
		\includegraphics[width=.40\textwidth]{current_capabilities.jpg}
		\caption{\textit{Graphic showing current platforms and how our would be better.}}
		\label{current_cap}
	\end{center}
\end{figurehere}

	
%%%%%%%%%%%%%%%%%%%%%%%%%%%%%%%%%%%%%%%%%%%%%%%%%%%%%%%%%%%%%%%%%%%%%%%%%%%%%%%%%%
\section{METHODS}

Much of the work done in the methodology of optimizing flight time was to find adequate models for the thrust, torque, and power required from the main rotor disks.

First model used in the beginning of the project used a simplified version of Momentum theory and actuator disk theory from \cite{helicopters2016}. This model took into account the number of rotors, the rotor radius, the disc loading of each rotor, and how much fuel was available onboard the aircraft. However, this model did not take into account the blade speed, blade pitch angle, or the number of blades per rotor. The model did, however, give us a baseline for initial optimization results.

Second model used to develop the optimization framework for this project was based on an MIT lecture given on aircraft propeller performance given in \cite{mit2016}. In order to test the theory of this model and also to further the progress of the optimization model, only two of the design variables were chosen to optimize: rotor radius and fuel consumption rate, $f_{rate}$. Flight time was then calculated by 
\begin{equation}
	t_{flight} = f_{capacity}/f_{rate}.
\end{equation}
This allowed for a very easy calculation in which to test the optimization technique. Without constraints, the solution to maximizing flight time would be trivial. To properly use the constraints, power produced ($P_{prod}$), power required ($P_{req}$), thrust produced ($T_{prod}$), and thrust required ($T_{req}$) all had to be calculated to make sure that the rotor radius had a relationship to the fuel consumption rate. These calculations used some approximations in the rotor dynamics. The following calculations come from a MIT lecture on thermodynamics.

In order to calculate thrust of each rotor blade, the thrust has to be calculated differentially along the blade because of the angular velocity. The differential thrust is given by 
\begin{equation}
	dT = dLcos(\phi + \alpha_i) - dDsin(\phi + \alpha_i)
\end{equation}

where $\phi = \tan^{-1}(\frac{V_i}{V_n})$, and $\alpha_i = \theta - \tan^{-1}(\frac{V_i}{V_n})$ and $\theta$ is the blade pitch angle (held constant in this case of the optimization). $dL$ and $dD$ are the differentials of the lift and drag, respectively, and are given by
\begin{equation}
	dL = \frac{1}{2}\rho V_e^2 c C_l dr
\end{equation}

and 

\begin{equation}
	dD = \frac{1}{2}\rho V_e^2 c C_d dr.
\end{equation}

The coefficients of lift and drag were held constant at $C_l = 0.5438$ and $C_d = 0.0141$ (obtained from a symmetric airfoil pitch at $\alpha = 5$ degrees). Splitting the rotor into 10 sections and integrating seemed to give a reasonable estimate to the thrust produced by each rotor. This was then constrained by making sure the thrust produced was greater than or equal to the thrust required, which was
\begin{equation}
	T_{req} = AUW\times g.
\end{equation}

In order to make sure that enough power was being produced by the engine (and prevent the fuel consumption rate to going to 0), the power produced was given by 
\begin{equation}
	P_{prod} = f_{rate}E_{\rho f} \eta_{engine}
\end{equation}

and was made sure that it was greater than or equal to the power required. To calculate the power required, the torque required to turn each rotor was calculated differentially by the equation
\begin{equation}
	dQ = r[dLsin(\phi + \alpha_i) + dDcos(\phi + \alpha_i)].
\end{equation}

The power required was then calculated by 
\begin{equation}
	P_{req} = \omega_rQ_{req}2n_{rotors}.
\end{equation}

This model did converge to an optimal solution when executed with gradient based optimization methods discussed in the next section. However, in this model, not all of the design variables were implemented due to the complexity these additional variables brought to the calculus in this method.

The third model used to try to represent the aerodynamics of each rotor blade was 

\begin{itemize}
	\item{Analysis models}
	\begin{itemize}
		\item{Empirical engine models}
		\item{Rotor Aerodynamic models}
	\end{itemize}
	\item{Simplifications - since we are stupid}
\end{itemize}

\section{OPTIMIZATION}
Setup the formal optimization problem
\begin{equation}
min. \quad f\of{x} = -flightTime\of{x}
\label{eq:objective}
\end{equation}
\begin{equation}
w.r.t. \quad x = [variables]^T
\label{eq:vars}
\end{equation}
\begin{equation}
s.t. \quad cons\of{x} <= 0 
\label{eq:constrants}
\end{equation}

What optimizers, setup, comparing matlab and python. 

Figure of Pareto fronts...

constraint sensitivity - aerodynamics, engine efficiencies...

\section{RESULTS AND DISCUSSION}

\begin{itemize}
	\item{Table showing the optimal design parameters}
\end{itemize}


%%%%%%%%%%%%%%%%%%%%%%%%%%%%%%%%%%%%%%%%%%%%%%%%%%%%%%%%%%%%%%%%%%%%%%%%%%%%%%%%%%

\section{CONCLUSION}

lots of really good conclusions

%%%%%%%%%%%%%%%%%%%%%%%%%%%%%%%%%%%%%%%%%%%%%%%%%%%%%%%%%%%%%%%%%%%%%%%%%%%%%%%%
% \section*{APPENDIX}

% Appendixes should appear before the acknowledgment.

% \section*{ACKNOWLEDGMENT}

% Important people/organizations who made it possible


%%%%%%%%%%%%%%%%%%%%%%%%%%%%%%%%%%%%%%%%%%%%%%%%%%%%%%%%%%%%%%%%%%%%%%%%%%%%%%%%

\bibliography{./library}
\bibliographystyle{ieeetr}



\end{document}


%%%%%%%%%%%%%%%%%%%%%%%%%%%%%%%%%%%%%%%%%%%%%%%%%%%%%%%%%%%%%%%%%%%%%%%%%%%%%

% SAVED STUFF






%new document




